\documentclass[a4paper,11pt]{article}
\usepackage[top=1in,bottom=1in,left=1.25in,right=1.25in]{geometry}
\linespread{1.25}

\title{SoLISP Design}
\author{Yuheng Kuang (kuangyuheng@gmail.com)}

\begin{document}
\maketitle

\section{Introduction}


\section{A taste of SoLISP}


\subsection{Hello World}

Say hello world using sys.stdout.write. It represent the LISP natual of SoLISP and how it integerate with Python.

\begin{verbatim}
(import sys)
(def (hello :name "World")
   (@ sys stdout write ((% "Hello %s!\n" name))))

(hello) ; => "Hello World!\n"
(hello "Yuheng") ; => "Hello Yuheng!\n"
\end{verbatim}


\subsection{Everything is expression}

As in LISP, Everything in SoLISP is expression.

\begin{verbatim}
(def (random_add :large False)
  (+ (if large 10000 else 1) 1))

(print (random_add)) ; => 1
(print (rnadom-add True)) ; => 10001
\end{verbatim}


\subsection{Powerful for}

A simple list comprehension example.

\begin{verbatim}
(print (for x <- [1 4 2 3]
          (if (> x 2) (cont)) (* x x)))
; => [1 4]
\end{verbatim}

A nested example.

\begin{verbatim}
(print
 (for x <- ["a" "b"]
    (emit* (for y <- [1 2] [x y]))) (cont)))
; => [["a" 1] ["a" 2] ["b" 1] ["b" 2)]
\end{verbatim}


\subsection{Restricted loop}

Looping in SoLISP is more restricted. You specify init value of the loop, then use \verb|(cont next_value)| to loop, the loop will break if you do nothing.

\begin{verbatim}
(print
 (loop total <- 1
       (if (< total 1000) (cont (* total 2)))))
; => 1024
\end{verbatim}

\subsection{Pattern matching}

The test\_proc is a one-argument function, that returns input when input is int and larger than 0, else raise an \verb|MatchException|.

\begin{verbatim}
(= test_proc (# x :int ?(> x 0)))
(test_proc 1)  ; => 1
(test_proc "1") ; => raise
(test_proc 0) ; => raise
\end{verbatim}

Pattern matching can be used in assignment.

\begin{verbatim}
(= [0 x . remain] src)
\end{verbatim}

SoLISP will check the value of src's first element (equals 0?), then assign second element to x and remaining elements to remain. It's called structual matching.

\subsection{Proc: try-match-do model}

Pattern matching is only a very limited case of the powerful Proc engine.

\begin{verbatim}
(= commander
 (# (str.split ["add" (int x) (int y)]) => (+ x y)
  # (str.split ["dec" (int x) (int y)]) => (- x y)
  # => -1))

(commander "dec 10 2") => 8
(commander "add 1 a") => -1
(commander "blahblahblah") => -1
\end{verbatim}

It also shows how SoLISP can match virtually everything. The first two section of the proc split input string into list, then matching it into a list pattern. It's called ``Extractor'', we can use extractor to convert any object into basic data structure and match them.

\subsection{Proc + Looping}

Proc can be used in ``\verb|for|'' and ``\verb|loop|''.

Concise version of list comprehension. 

\begin{verbatim}
(for (# x ?(<= x 2) => (* x x)) <- [4 1 3 2])  ; => [1 4]
\end{verbatim}

\begin{verbatim}
(loop (# x ?(< x 1000)) <- 1 (cont (* x 2)))
\end{verbatim}

\subsection{Macros}

We use ``match'' expression everywhere.

\begin{verbatim}
(match x
 # ?(< _ 0) => "<"
 # ?(== _ 0) => "=="
 # => ">")
\end{verbatim}

This expression is entirely implemented in SoLISP, by defining a macro.

\begin{verbatim}
(= macro_match (# ['match value . proc] => `('= proc value)))
\end{verbatim}

It translates \verb|(match value . proc)| into \verb|(= proc value)|, then assignment primitive do the job.

\subsection{Full example}

For full example, see Appendix \ref{full_example}.

\section{Data type}

\section{Basic syntax}


\section{Pattern matching}


\section{Looping and iterating}


\section{Customize SoLISP}


\begin{appendix}

\section{Full example: Generating DOM Tree}
\label{full_example}

\end{appendix}
\end{document}
